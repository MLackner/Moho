
% This LaTeX was auto-generated from MATLAB code.
% To make changes, update the MATLAB code and republish this document.

\documentclass{article}
\usepackage{graphicx}
\usepackage{color}

\sloppy
\definecolor{lightgray}{gray}{0.5}
\setlength{\parindent}{0pt}

\begin{document}

    
    \begin{verbatim}
function P = heatingRateFS_Vac( m,t )
% Calculates the dissipated power in the sample at a given time.
%
%   Input:
%   m   =   Mesh
%   t   =   Current simulation time in s
%
%   Output:
%   P   =   Total dissipated power in the sample in W
%

% Set maximum voltages for each heating cycle in volt
Umax = [3.0 4.5 6.0 7.5];

% Set start and end times for the heating cycles in seconds
ton = [44 234 498 870];
toff = [190 398 707 1231];

% Set the applied base voltage in volt
U0 = 0.1;

% Set voltage ramp in V/s
ramp = 0.1;


% Get the mean temperature of ITO film (Supposed to be the temperature of the
% layer in which the energy is induced) and calculate its resistance

% Get matrix indices of the heat source elements
idx = m.source.Heat;
% Get temperatures of the heat source elements
Temperatures = m.temperature(idx);
% Calculate the mean temperature of all these elements
T = mean( Temperatures(:) );

% Approximate temperature dependece of the ITO film resistance
R_ITO = 1/100*T + 14;


% Calculate applied voltage
for i=1:numel( Umax )

    if t >= ton(i) && t <= toff(i)
        % Sample is heated
        % Calculated applied voltage drop based on the used ramp
        U = (t - ton(i))*ramp + U0;

        if U > Umax(i)
            % Voltage based on ramp is higher than maximum voltage.
            % Set voltage to maximum voltage
            U = Umax(i);
        end

        % Applied voltage found. Leave the for loop.
        break

    else
        % Sample is not heated.
        % Use base voltage
        U = U0;
    end

end

% Calculate dissipated power in W
P = U^2/R_ITO;

end
\end{verbatim}

        \color{lightgray} \begin{verbatim}Error using heatingRateFS_Vac (line 30)
Not enough input arguments.\end{verbatim} \color{black}
    


\end{document}
    
